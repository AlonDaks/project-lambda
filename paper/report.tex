\documentclass[11pt]{article}

\usepackage[margin=0.75in]{geometry}

\title{fMRI Dataset from Complex Natural Simulation with Forrest Gump: A Restudy}
\author{
  Chang, Jordeen\\
  \texttt{jodreen}
  \and
  Daks, Alon\\
  \texttt{AlonDaks}
  \and
  Luo, Ying\\
  \texttt{yingtluo}
  \and
  Yu, Lisa Ann\\
  \texttt{lisaannyu}
}

\bibliographystyle{siam}

\begin{document}
\maketitle  

\abstract{Most fMRI studies use highly simplified stimulus that are vastly dissimilar
from what people experience in everyday life. This study sought to create
a dataset of naturally occurring brain states by exposing participants
to a more complex stimulus, the audio description of Forrest Gump. This particular
audio description allows for the study of auditory attention and cognition, 
language and music perception, and retrieval of explicit memory without the
effect of visual imagery. In addition, this dataset uses inter-individual 
synchronicity to study responses to complex processing. Originally,
representational similarity analysis was used to identify similar patterns
across brains.}

\section{Introduction}

The main purpose of the original study was to examine properties of brain response
patterns that are supposedly common when people are exposed to audio and movie
simulation. We intend to replicate their experiment using the data they gathered
from the 20 participants. For example, a BOLD time-series similarity measure
(e.g. correlation) is often used to quantify similarities in responses among
individuals. Hank et al. recognized that this was a common approach, but they
went beyond that and also implemented representational similarity analysis 
(RSA). To do so, we will create dissimilarity matrices for 18 individuals using
the same searchlight mapping approach that they used (Subjects 4 and 10 were not
included due to missing data). Doing so will capture 2nd-order isomorphisms in
response patterns. Lastly, to access statistical significance, we will transform
the representational consistency map into percent rank with respect to the total
distribution of the DSM correlations. We'll calculate the mean correlation
coefficient and compare our value to theirs.

Before we formally began, we performed basic sanity checks on the data. We
have downloaded and loaded the files successfully, and we have confirmed that
we have data from all subjects. Reproducibility is crucial in research, especially
when such high volumes of data are involved, because it allows other people to
fact check the work. When people collaborate, new insights can be shed and the
rate of progress is expedited. For this study, we will begin by following the 
exact steps Hanke and his team took. Along the way, when we have identified areas
that they did not have time to thoroughly research, we will then delve deeper in
an effort to shed more insights.

Identify a published fMRI paper and the accompanying data
\cite{hank2014audiomovie}.

\section{Data}
The data is curated and segmented into 20 .TGZ files, where each of the 20 .TGZ 
files corresponds to one of the 20 subjects in the experiment. Each subject 
accounts for approximately 16 GBs of data. We verified the usability of the data
by inspecting and loading data corresponding to subject 1. We limited our 
initial exploration to a single subject since downloading each .TGZ takes 
approximately one hour. To ensure speedy access to the overall dataset when we 
begin central project work, our strategy for getting all the data entails each 
group member spending five hours downloading a different quarter of the overall
dataset, and then locally transferring the remaining three-quarters of the data
from our harddrives. Each subject's data includes several formats: subject
metadata(CSV), Raw BOLD functional MRI, Raw BOLD functional MRI 
(with applied distortion correction), Raw BOLD functional MRI (linear anatomical
alignment), Raw BOLD functional MRI (non-linear anatomical alignment), along 
with several Structural MRI datasets. The corrected and aligned versions of the
data attempt to eliminate device and scan related noise. Scan data is 
accessible in nibabel compatible formats (.NII).

\section{Methods}
\section{Results}
\section{Discussion}


\bibliography{project}

\end{document}
