\documentclass[11pt]{article}

\usepackage[margin=0.75in]{geometry}

\title{fMRI Dataset from Complex Natural Stimulation with Forrest Gump: A Restudy}
\author{
  Chang, Jordeen\\
  \texttt{jodreen}
  \and
  Daks, Alon\\
  \texttt{AlonDaks}
  \and
  Luo, Ying\\
  \texttt{yingtluo}
  \and
  Yu, Lisa Ann\\
  \texttt{lisaannyu}
}

\bibliographystyle{siam}

\begin{document}
\maketitle

\abstract{Most fMRI studies use highly simplified stimulus that are vastly
dissimilar from what people experience in everyday life. This study sought to
create a dataset of naturally occurring brain states by exposing participants
to a more complex stimulus, the audio description of Forrest Gump. This
particular audio description allows for the study of auditory attention and
cognition, language and music perception, and retrieval of explicit memory
without the effect of visual imagery. In addition, this dataset uses inter-
individual synchronicity to study responses to complex processing. Originally,
representational similarity analysis was used to identify similar patterns
across brains.}

\section{Introduction}

The main purpose of the original study was to examine properties of brain
response patterns that are supposedly common when people are exposed to audio
and movie stimulation. We intend to replicate their experiment using the data
they gathered from the 20 participants. For example, a BOLD time-series
similarity measure (e.g. correlation) is often used to quantify similarities
in responses among individuals. Hank et al. recognized that this was a common
approach, but they went beyond that and also implemented representational
similarity analysis (RSA). To do so, we will create dissimilarity matrices for
18 individuals using the same searchlight mapping approach that they used
(Subjects 4 and 10 were not included due to missing data). Doing so will
capture 2nd-order isomorphisms in response patterns. Lastly, to access
statistical significance, we will transform the representational consistency
map into percent rank with respect to the total distribution of the DSM
correlations. We'll calculate the mean correlation coefficient and compare our
value to theirs.

Before we formally began, we performed basic sanity checks on the data. We
have downloaded and loaded the files successfully, and we have confirmed that
we have data for a test subject. Reproducibility is crucial in research,
especially when such high volumes of data are involved, because it allows
other people to fact check the work. When people collaborate, new insights can
be shed and the rate of progress is expedited. For this study, we will begin
by extracting data we see fitting, and proposing interesting questions that we
will attempt to answer.

Identify a published fMRI paper and the accompanying data
\cite{hank2014audiomovie}.

\section{Data}

The data is curated and segmented into 20 .TGZ files, where each of the 20
.TGZ files corresponds to one of the 20 subjects in the experiment. Each
subject accounts for approximately 16 GBs of data. We verified the usability
of the data by inspecting and loading data corresponding to subject 1. We
limited our initial exploration to a single subject since downloading each
.TGZ takes approximately one hour. The download for each subject includes a
lot of other information besides just the fMRI data, such as cardiac and
respiratory trace, angiographies, and structural MRI data. Each subject's fMRI
data includes several formats: Raw BOLD functional MRI, Raw BOLD functional
MRI (with applied distortion correction), Raw BOLD functional MRI (linear
anatomical alignment), and Raw BOLD functional MRI (non-linear anatomical
alignment). The corrected and aligned versions of the data attempt to
eliminate device and scan related noise. Scan data is accessible in nibabel
compatible formats (.NII).

\section{Methods}

\subsection{Extracting Data}

We will not be using all of the data. The data
files are very large: for each subject, the download size is approximately 16
GB, and that does not even include unzipping all the file inside. Given that
we have twenty subjects, downloading all of this could be a paper in it of
itself. Since the focus of this class is mainly fMRI data, we decided to
ignore everything else. However, the fMRI data is still not of a trivial size
because three versions of the data are included: the raw data, the linear
alignment, and the non-linear alignment. We will only be choosing one of them
(the one linearly aligned) because using all of them would just be redundant.

We will be only working with the first subject for now for testing purposes,
but if time allows, we will run our code on a couple of other subjects for
comparison testing. And to ensure speedy access to other subjects' datasets,
our strategy for getting all the data entails each group member spending
downloading a different quarter of the overall dataset, and then locally
transferring the remaining three-quarters of the data from our hard drives.

\subsection{Analysis}

After perusing all the stimulus related data (e.g. data
annotating each Forest Gump scene and when a subject was exposed to that
scene), we decided to guide part of our analysis to see if we could summarize
any interesting trends in fMRI response with respect to these features.
Additionally, by investigating a single subject for our initial analysis and
using the linearly aligned transformed data, we have limited our project’s
scope to approximately 5 GBs from the roughly 320 GBs that were published with
the paper. With the data reduced to a manageable state, we have successfully
overcome this obstacle. Finally, by introducing a data\_path.json file to
reference where each raw data file is located, we have designed a clean schema
for our python scripts to find and load data in the project repository.

We are deviating from the original data analysis. The paper used the raw data
which varied largely with every subject, and processed that by standardizing
among twenty of them with both linear alignments and non-linear alignments. We
decided that we would simply use the already processed linearly aligned data
since they already provided it, and we would not be able to do it as well as
them.

Our initial plan is to first conduct basic exploratory data analysis
and then further delve into the findings that strike us as intriguing or
unusual. We began by writing functions that loaded the .nii files for each
subject and calculated and plotted the standard deviations across voxels. From
there, we will continue our analysis by isolating outliers and referencing
those data points with the corresponding movie time to look for correlations
between certain movie scenes and certain physiological responses.

After exploring the files provided for Forrest Gump, we saw that there exists
a csv file of the scenes in the movie with a time-stamp, a brief scene
description, whether it takes place in day or night, and if it is inside or
outside. We will try to see if there is anything interesting in the data based
on these characteristics of the movie scenes.

After determining which points in time were during the day and which were
during the night, and which were inside and which were outside, we can perform
a t-test to determine if the signal is significantly different between each of
the two groups for each voxel.  We will perform a multiple comparisons test to
correct for the number of t-tests we will be running. We will also model
each of those voxels that are statistically significant to see what their time
courses look like.

We have a few ideas as to what questions to ask, but are in the process of
looking at other ideas. There is an entire website dedicated to studying this
dataset, studyforrest.org, and we are researching to see what other people
have done.


\section{Results}

So far, we have written functions that plot the standard
deviations across voxels for individual subjects using Matlibplot. In
addition, we also plan on exploring possible differences in physiological
responses to certain movie scenes by gender and age. Given that we would have
to comb through around 16 GB worth of data just for each subject, we will
initially narrow our analysis to just one representative from each group (e.g.
one female and one male). And if time permits, we will run the our analysis on
all 20 subjects (e.g. 8 females and 12 males).


\section{Discussion}
\subsection{Strengths}

Making our project reproducible is among our strengths. Our team has a firm
understanding of command line and python scripts, along with the importance of
grouping common behavior in general functions, such as not merging new
functions into master unless those functions are unit tested, to ensure that
a) functions work according to specified behavior and b) future changes do not
break functions that had been working.

Learning how to work with the data in class was helpful because we have never
worked with images in that type of format. We were not familiar with libraries
like nibabel and thus it was very helpful to see how we could use them in our
analyses. Exercises working with other fMRI data and various calculations we
could perform on them was informative and relevant.


\subsection{Obstacles}

The main objective of the original study and paper was to gather raw data,
mostly for other researchers to use in their own studies. Consequently, we
were not given much direction in terms of what analyses we should initially
conduct or reproduce. Also, as previously mentioned, the sheer amount of data
we have to work with is almost overwhelming, especially given our limited
computing resources. However, we assuaged this issue by using external hard
drives to transfer data among group members.

We have differing prior experiences with Github and with research.  Those who
have more experience with git and Github help those who have less experience
set up repos and push and pull correctly.  Finding a method of communication
was another issue we initially faced: we have been communicating via Facebook,
and switching to communicate on Github to directly reference code has been
difficult.  As a result, many issues on Github and pull requests have not been
seen by other members.  We have addressed this issue by reminding one another
to look at pull requests on Facebook. We have varying areas of expertise, such
as code review and statistical analysis, but we have used that to our
advantage by having group members work on their areas of expertise.

Understanding the data - specifically in terms distinguishing the several
forms of normalized data - and what conclusions we can draw is difficult, but
we understand that that is beyond the scope of our project. Although our main
goal is to become familiar with reproducibility and collaboration via the
software tools taught in class, feeling shaky on understanding the statistical
meaning of our results is disconcerting. For example, a deeper understanding
of time series would enable us to perform stronger statistical techniques.

To successfully complete our project we need to keep the data we use well
structured and organized, perform insightfully and interpretable exploratory
data analysis, and perform hypothesis tests referencing Forest Gump scene
features (although these tests are not expected to indicate causal
inferences). Finally success also depends on doing these items in a
reproducible and collaborative manner. We must strive to automate our analysis
routines through command line interfaces and exploit the modern git and github
workflow.


\bibliography{project}

\end{document}
